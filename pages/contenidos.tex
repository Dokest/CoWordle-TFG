\tutor{El título de esta sección es muy raro. Renombrala 'Tecnologı́as, Herramientas y Metodologı́as', distingue entre tecnologı́as y herramientas.}

\section{Tecnologías}


\subsection{Typescript}
Typescript \cite{typescript_docs} es un lenguaje desarrollado por Microsoft que se compila a Javascript. Las principales funcionalidades que añade Typescript son: un robusto sistema de tipados y un completo sistema de genéricos.

Typescript reduce los tiempos de desarrollo permitiendo detectar errores de tipados mientras programas, además, ayuda al desarrollador utilizando autocompletado sobre miembros de clases, variables y funciones.

\subsubsection{¿Cómo se ha utilizado Typescript?}
Typescript ha sido el único lenguaje de programación utilizado en toda la práctica. Gracias a los avances de Node/Deno, Typescript se ha podido utilizar tanto en el front-end, como en el back-end.

Además, en el trabajo se han creado sistemas específicos de ayuda sobre los tipados, así como para la auto compleción de, por ejemplo, las llamadas de websockets.


\subsection{Node.js}
Node \cite{nodejs_docs} es probablemente una de las creaciones más influyentes en el mundo web, creada por Ryan Dahl en 2009, \tutor{después de una , no encaja un 'y', prueba con un 'que'}y permite ejecutar código Javascript en el lado del servidor, sin necesidad de usar un navegador.

Node utiliza internamente el motor de Javascript (y WebAssembly) V8 desarrollado para Chromium por Google.

Una de las principales diferencias entre Node.js y otros sistemas como el motor de PHP, es que no es necesario levantar un proceso para cada llamada, en su lugar, Node.js es capaz de ir recibiendo llamadas e irlas respondiendo de manera concurrente\tutor{Node no es "concurrente" al uso, funciona con un event-loop (es uni-hilo), sigue una arquitectura orientada a eventos, echa un ojo aquí: \url{https://nodejs.org/en/docs/guides/event-loop-timers-and-nexttick}o aquí \url{https://tsh.io/blog/simple-guide-concurrency-node-js/}}, esto le permite responder mucho más rápido y eficientemente que otros sistemas.

En 2022 y según la encuesta de StackOverflow \cite{so_survey} en la que participaron más de cuarenta y cinco mil personas, Node.js es la tecnología web más usada entre desarrolladores profesionales siendo usada por un 46.31\% de estos.

\subsubsection{¿Cómo se ha utilizado Node.js?}
Aunque Node.js no se ha utilizado directamente en la práctica, sí que ha permitido indirectamente que el resto de herramientas hayan podido ser utilizadas, especialmente SvelteKit.


\subsection{Deno}
Deno \cite{deno_docs} es una tecnología open-source alternativa a Node, creada por el mismo desarrollador que creó Node.js, Ryan Dahl, que presentó Deno en una charla llamada “10 things I Regret About Node.js” en la conferencia de Javascript que se estaba celebrando en Europa.

Los principales cambios respecto a Node.js son la incorporación de Typescript sin herramientas adicionales, y la incorporación de un sistema de permisos muy estricto que impide la ejecución de código malicioso al usar librerías externas.

Además, Deno incluye mejoras como el manejo automático de dependencias, una instalación muy rápida, un corredor de tests interno así como otro de benchmarks, o la simplicidad de uso al no necesitar herramientas externas como linters o formatters, que ya vienen incluidos con la propia instalación de Deno.

Deno es capaz de ejecutar código Typescript directamente utilizando un compilador internamente, sin necesidad de generar archivos Javascript por parte del desarrollador, y sin generar archivos en el proyecto resultado de la compilación.

\subsubsection{¿Cómo se ha utilizado Deno?}
Deno no estaba planteado como herramienta cuando se comenzo el trabajo, pero despues de necesitar hacer tests para la parte de servidor de WebSockets, y buscar posibles herramientas de testeo para Node.js, surgió la idea de cambiar esta parte de Node.js por Deno.

Como ya se ha comentado, Deno cuenta con su propio corredor\tutor{mejor 'ejecutor'} de tests, y además, es casi completamente compatible con Node.js, migrar la aplicación fue un trabajo de un par de horas, en las que simplemente hubo que cambiar la forma de inicializar la aplicación para ajustarse con el sistema de Deno.

\subsubsection{Curiosidad, el nombre de Deno}
Como el creador de Deno es también el creador de Node.js, Ryan Dahl de manera cómica eligió el nombre de Deno invirtiendo las sílabas de Node (No-de -> De-no).


\subsection{SvelteKit}
SvelteKit \cite{sveltekit_docs} es una tecnología para el desarrollo front-end y back-end desarrollada por Rich Harris.

SvelteKit es la unión de dos tecnologías, por una lado Svelte, que es un framework de desarrollo front-end, alternativo a React, Angular o Vue. Y por otro lado, la parte del nombre Kit viene de la incorporación a Svelte de un sistema de routing y de poder ejecutar código en back-end, también habilita la posibilidad de hacer SSR (Server Side Rendering), al igual que Next permite hacer SSR para React, o Nuxt permite hacer SSR para Vue.
\tutor{Esta frase ha quedado muy larga, dividela o trata de utilizar otros conectores}

\subsubsection{Svelte}
Svelte \cite{svelte_docs} es un compilador de componentes web que permite al desarrollador programar la vista, ejecución y estilado de un componente en un solo archivo, y tras ser compilado, devuelve código que manipula el DOM y es reactivo a cambios en la información presentada.

Svelte, a diferencia de React o Vue, no necesita de un framework en runtime para funcionar, el compilador devuelve el código Javascript necesario para mostrar la información necesaria y manejar la reactividad, sin necesidad de un shadow-DOM, esto disminuye la cantidad de código que hay que enviar al cliente en la petición inicial, lo que mejora el rendimiento y disminuye el “time to interactive”.

Además, la sintaxis de Svelte es muy simple y muy cómoda de usar. Svelte incluye también una librería de animaciones de componentes, y un sistema para manejar la reactividad que simplifica el desarrollo utilizando una sintaxis sencilla para hacer que cualquier función o variable sea reactiva.

\subsubsection{SvelteKit}
SvelteKit añade a Svelte la posibilidad de realizar operaciones en la parte de servidor, permitiendo hacer SSR, que permite mejorar reducir el time to interactive enviando al cliente solo el HTML resultante de realizar el renderizado en el servidor, y otras operaciones.

SvelteKit es considerada como la manera más sencilla y rápida de desarrollar una aplicación de Svelte, y por esa razón se ha utilizado en este trabajo.

\subsubsection{¿Cómo se ha utilizado SvelteKit?}
SvelteKit ha fundamentado la base que ha permitido añadir reactividad a la aplicación de manera sencilla, su cómoda sintaxis facilita el trabajar con los datos, especialmente aquellos que requieren de reactividad.

\subsubsection{Tailwind}
Tailwind \cite{tailwind_docs} es un framework de CSS que permite utilizar la mayor parte del poder de CSS directamente desde la plantilla HTML, además, es compatible con el sistema de componentes de Svelte.

Tailwind funciona utilizando clases HTML, que tienen una representación en el framework de CSS, esto consigue que con muy poquito esfuerzo tengamos una página lista.

Tailwind no es una librería de componentes ya hechos, como lo puede ser Bootstrap, Tailwind solo incluye código CSS que es el que será interpretado por el navegador.

\subsubsection{¿Cómo se ha usado Tailwind?}
Tailwind se ha usado para la mayoría del estilaje \tutor{mejor: se ha usado para la maquetación}de la aplicación, aunque ha habido pequeñas partes que si han necesitado de su propio CSS para crear estilos más complejos.

\subsection{Git y GitHub}
Git es un sistema de control de versiones distribuido, que permite a desarrolladores sincronizar su código entre versiones.

GitHub es el servicio web que ha almacenado el código, es el servicio de Git más grande e importante.

\subsection{Docker y DockerHub}
Docker es una plataforma para desarrollar y desplegar aplicaciones de todo tipo, permite separar las aplicaciones de la infraestructura reduciendo los tiempos de desarrollo.

Cada aplicación de Docker se puede subir a DockerHub, la plataforma para compartir aplicaciones Docker oficial, a través de esta plataforma es posible desplegar tu aplicación en cualquier computadora con tan solo saber el nombre del repositorio y las credenciales (en caso de ser necesarias).

\subsection{Playwraight}
Playwraight \cite{playwright_docs} es el framework para testear componentes recomendado por SvelteKit, es un software muy capaz que permite con muy poco trabajo y con una sintaxis muy sencilla, hacer testeo determinista sobre los componentes web. En numerosos frameworks de testeo de páginas web hay problemas para saber cuando la página web ha cargado los componentes, Playwraight provee de sistemas asíncronos (basados en Promesas) para esperar a que los componentes hayan cargado sin complicar el código.

\subsubsection{¿Cómo se ha usado Playwraight?}
Playwraight se ha usado para hacer tests para componentes, y parcialmente para flujos. Hacer tests de flujos completos ha sido más complejo, ya que es difícil interceptar las comunicaciones de Websockets.

\section{Metodología}
Este trabajo ha seguido una metodología iterativa e incremental en espiral, en la cual el tutor y el alumno se reúnen de manera periódica y hacen revisión de objetivos y progreso.

En cada reunión se hacen propuestas de mejora, y estas se convierten en objetivos para la siguiente revisión. Poco a poco el desarrollo del trabajo se fue materializando, empezando con los pasos más básicos como tener una plantilla sobre la que poder ir probando desarrollos, pasando por el aprendizaje de la tecnología de WebSockets, como el arreglo de bugs que han ido surgiendo a lo largo del tiempo.

La metodología escogida ha funcionado muy bien para el tipo de proyecto, la progresión iterativa y en fragmentos ha permitido que la práctica\tutor{TFG o trabajo, no práctica} se haya desarrollado lentamente, y con la atención puesta en cada fragmento de desarrollo, sin llegar a encontrarse con un gran panel de tareas que pueda resultar abrumador, además, el planteamiento a futuro ha permitido que la aplicación haya crecido adecuadamente, sin necesidad de realizar grandes cambios.
