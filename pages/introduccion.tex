Se puede añadir texto antes de empezar la primera sección.


\section{Contexto y alcance}

Contexto. Situar al lector. Objetivo general y alcance del trabajo.


\section{Estructura del documento}

La estructura del TFG no es fija. El tutor indicará una estructura adecuada dependiendo del trabajo concreto.\tutor{Comentario del tutor}

Se puede incluir dentro de cada apartado secciones adicionales. La copia en papel de la memoria del TFG será encuadernada en pasta dura de color azul (p.e. encuadernación tipo chanel). La portada, que puede ser una pegatina transparente, seguirá el modelo que se adjunta, que incluye el escudo y nombre de la URJC, la titulación cursada por el alumno, el curso académico, el título del TFG, el autor y el o los directores/tutores.\alumno{Comentario del alumno}


\subsection{Trabajos de grados en informática}

Una posible estructura de la memoria final asociada con cada TFG podría ser la siguiente (leed la normativa de TFG):
\begin{enumerate}
 \item Introducción
 \item Objetivos (incluyendo descripción del problema, estudio de alternativas y metodología empleada)
 \item Descripción informática (puede incluir especificación, diseño, implementación y pruebas).
 \item Experimentos / validación
 \item Conclusiones (incluyendo los logros principales alcanzados y posibles trabajos futuros)
 \item Bibliografía
 \item Apéndices
\end{enumerate}


\subsection{Trabajos del grado en matemáticas}

Una posible estructura de la memoria final asociada con cada TFG podría ser la siguiente:
\begin{enumerate}
 \item Introducción
 \item Objetivos (incluyendo descripción del problema, estudio de alternativas y metodología empleada)
 \item Material y métodos / Metodología / Cuerpo del trabajo (describir las metodologías empleadas en el desarrollo del TFG o el desarrollo del mismo en caso de ser un trabajo de recopilación bibliográfica sobre un tema).
 \item Resultados (opcional, dependiendo del tipo de trabajo desarrollado)
 \item Conclusiones (incluyendo los logros principales alcanzados y posibles trabajos futuros)
 \item Bibliografía
 \item Apéndices
\end{enumerate}